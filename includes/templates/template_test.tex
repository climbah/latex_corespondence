\documentclass[10pt]{letter}
\usepackage{ifthen}
\usepackage[T1]{fontenc}
\usepackage[utf8]{inputenc}
\usepackage{graphicx}
\usepackage{fancyhdr}
\usepackage{business}
%set option address left
\setboolean{addressleft}{true}
\sname{Datahouse AG}
\saddress{Technoparkstrasse 1}
\splz{8005 Zürich}
\stel{078 771 55 22}
\sfax{078 771 55 23}
\semail{daenu.schmid@gmail.com}
\sweb{http://www.datahouse.ch}
\image{datahouse}
\setboolean{backaddress}{true}
\back{Datahouse AG}{Technoparkstrasse 1}{8005 Zürich}
\date{Schmitten, den 01. August 2010 }
\begin{document}
\begin{letter}{Daniel Schmid\\ Pergolastrasse 4 \\ 3185 Schmitten}

\subject{Ich bin ein Betreff und kann unter Umständen sehr lange sein. Was wird wohl passieren wenn Ich umgebroche werden muss?\\}
\opening{Sehr geehrte Dame, \\
Sehr geehrter Herr,\\}

Ein Schneider lebt mit seinen drei Söhnen und einer Ziege zusammen, die sie mit ihrer Milch ernährt, wozu sie täglich auf die Weide muss und dort die allerbesten Kräuter fressen darf. Als der Älteste sie schön geweidet hat und fragt, ob sie satt sei, antwortet diese: „Ich bin so satt, ich mag kein Blatt: meh! meh!“ Als aber der Vater zu Hause die Ziege fragt, antwortet sie mit einer Lüge: „Wo von sollt ich satt sein? Ich sprang nur über Gräbelein, und fand kein einzig Blättelein: mäh! mäh!“ Der Vater erkennt die Täuschung der Ziege nicht und jagt im Affekt den Ältesten mit der Elle aus dem Haus. Den beiden anderen Söhnen ergeht es die folgenden Tage genauso. Als der Vater die Ziege dann selbst ausführt und sie draußen so und zu Haus so antwortet, erkennt er, dass er seinen Söhnen Unrecht getan hat, schert der Ziege den Kopf und jagt sie mit der Peitsche fort. Die Söhne gehen bei einem Schreiner, einem Müller und einem Drechsler in die Lehre. Am Ende bekommt der Älteste einen unscheinbaren kleinen Tisch mit; wenn man zu dem sagt „Tischchen, deck dich!“, dann ist er sauber gedeckt und mit den herrlichsten Speisen versehen. Der Mittlere bekommt einen Esel; wenn man zu dem sagt „Bricklebrit!“, dann fallen vorne und hinten Goldstücke heraus. Alle drei Söhne verzeihen dem Vater schließlich während ihrer Wanderjahre und sehen die Möglichkeit, dass auch ihr Vater seinen Groll vergisst, sobald sie ihn mit ihrem eigenen Wunderding gewonnen haben. Die beiden älteren werden aber vor ihrer Heimkunft in ihrer Freigiebigkeit nacheinander vom gleichen Wirt betrogen, als der dem einen ein falsches Tischchen und dem anderen einen anderen Esel unterschiebt. Sie bemerken es erst, als sie ihr Wunderding zu Hause vorführen wollen. Sie schämen sich vor allen Gästen, die der Vater eingeladen hat, der jetzt weiter als Schneider arbeiten muss. Der Jüngste bekommt von seinem Meister einen Knüppel im Sack, der jeden Gegner verdrischt, wenn man sagt „Knüppel, aus dem Sack!“ und erst aufhört, wenn man sagt „Knüppel, in den Sack!“. Damit nimmt er dem Wirt das Tischchen und den Esel wieder ab, als der ihm den Sack – verwendet als Kopfkissen – stehlen will, dessen Wert er ihm vorher gepriesen hatte. Die Ziege verkriecht sich aus Scham über ihren kahlen Kopf in einen Fuchsbau, wo der Fuchs und dann der Bär vor ihren glühenden Augen erschrecken. Aber die Biene sticht ihr in den geschorenen Kopf, sodass die Ziege vor Schmerz flieht und, nun endgültig heimatlos geworden, abgängig ist.
Ein Schneider lebt mit seinen drei Söhnen und einer Ziege zusammen, die sie mit ihrer Milch ernährt, wozu sie täglich auf die Weide muss und dort die allerbesten Kräuter fressen darf. Als der Älteste sie schön geweidet hat und fragt, ob sie satt sei, antwortet diese: „Ich bin so satt, ich mag kein Blatt: meh! meh!“ Als aber der Vater zu Hause die Ziege fragt, antwortet sie mit einer Lüge: „Wo von sollt ich satt sein? Ich sprang nur über Gräbelein, und fand kein einzig Blättelein: mäh! mäh!“ Der Vater erkennt die Täuschung der Ziege nicht und jagt im Affekt den Ältesten mit der Elle aus dem Haus. Den beiden anderen Söhnen ergeht es die folgenden Tage genauso. Als der Vater die Ziege dann selbst ausführt und sie draußen so und zu Haus so antwortet, erkennt er, dass er seinen Söhnen Unrecht getan hat, schert der Ziege den Kopf und jagt sie mit der Peitsche fort. Die Söhne gehen bei einem Schreiner, einem Müller und einem Drechsler in die Lehre. Am Ende bekommt der Älteste einen unscheinbaren kleinen Tisch mit; wenn man zu dem sagt „Tischchen, deck dich!“, dann ist er sauber gedeckt und mit den herrlichsten Speisen versehen. Der Mittlere bekommt einen Esel; wenn man zu dem sagt „Bricklebrit!“, dann fallen vorne und hinten Goldstücke heraus. Alle drei Söhne verzeihen dem Vater schließlich während ihrer Wanderjahre und sehen die Möglichkeit, dass auch ihr Vater seinen Groll vergisst, sobald sie ihn mit ihrem eigenen Wunderding gewonnen haben. Die beiden älteren werden aber vor ihrer Heimkunft in ihrer Freigiebigkeit nacheinander vom gleichen Wirt betrogen, als der dem einen ein falsches Tischchen und dem anderen einen anderen Esel unterschiebt. Sie bemerken es erst, als sie ihr Wunderding zu Hause vorführen wollen. Sie schämen sich vor allen Gästen, die der Vater eingeladen hat, der jetzt weiter als Schneider arbeiten muss. Der Jüngste bekommt von seinem Meister einen Knüppel im Sack, der jeden Gegner verdrischt, wenn man sagt „Knüppel, aus dem Sack!“ und erst aufhört, wenn man sagt „Knüppel, in den Sack!“. Damit nimmt er dem Wirt das Tischchen und den Esel wieder ab, als der ihm den Sack – verwendet als Kopfkissen – stehlen will, dessen Wert er ihm vorher gepriesen hatte. Die Ziege verkriecht sich aus Scham über ihren kahlen Kopf in einen Fuchsbau, wo der Fuchs und dann der Bär vor ihren glühenden Augen erschrecken. Aber die Biene sticht ihr in den geschorenen Kopf, sodass die Ziege vor Schmerz flieht und, nun endgültig heimatlos geworden, abgängig ist.
Nun kann der König nicht anders und muss dem armen Schneider seine Tochter und das Königreich geben.
\vspace{20mm}
\closing{Mit freundlichen Grüssen}
Daniel Schmid
\end{letter}
\end{document}